\documentclass[../main.tex]{subfiles}

\begin{document}

\section{Introduction} \label{sec:intro}
Development in methods for solving integrals have been important in order to solve problems with an increasing degree of complexety. Guassian quadrature is a good example which is a method first developed by Jacobi in 1676. The first version gave exact results for algebraic polynomials of negree n-1 or less. The "new" Guassian version has a significant increase in accuaracy with exact results for polynomials of degree 2n-1 or less due to free choise of weights.

Gauss-Legendre and Gauss-Laguerre are two types of Gaussian quadrature which in this report will be compared in accuracy and speed for a multidimensional integral describing the energy of electrons in a Helium atom. In addition, two approaches to the Monte Carlo method of integration are implemented and compared as well. Parallelization will also be done to the program running the Monte Carlo integration.

First theory for the different methods is  presented, followed by our results and finally a discussions.

\end{document}
