\documentclass[../main.tex]{subfiles}

\begin{document}
\section{Introduction}\label{introduction}
The Velocity Verlet method is a widely used method for solving coupled ordinary differential equations. In this report, we will model our solar system's dynamics, utilizing said method. The equations to solve, simply come from Newton's laws of motion in gravitational fields, although we will make a small modification down the road to account for relativistic effects and achieve greater accuracy. Due to the sheer number of variables and methods required to calculate the motion of this many bodies, we will object orient our code - simplifying the process of adding bodies and making them interact with each other. \\
On our way to the final model of the solar system, we are going to explore the accuracy-differences between the Velocity Verlet and the Euler Forward method. The impact the massive planet Jupiter has on Earth's orbit is explored in isolation with the sun, in addition to how Mercury's perihelion precesses when relativistic effects are accounted for. Closing everything with the final model of our solar system, we study the stability of our program at different time steps, and check that the total energy and angular momentum is conserved - making it physically accurate. \\
The report has a theory part explaining the physical theory and the thought behind our computational implementation of it. Following this are our results and finally a discussion of them. 
% Might lack some parts. Just hmu on Messenger if something needs to be added.
\end{document}
