\documentclass[../main.tex]{subfiles}

\begin{document}

\section{Introduction} \label{sec:intro}
Development in methods for solving integrals has been important in order to solve problems with a increasing degree of complexety. Guassian quadrature is a good example which is a method first developed by Jacobi in 1676. The first version gave exact results for algebraic polynomials of negree n-1 or less. The "new" Guassian version has a significant increase in accuaracy with exact results for polynomials of degree 2n-1 or less due to free choise of weights.


KILDE:
https://www.jstor.org/stable/24898684?seq=2#metadata_info_tab_contents

Gauss-Legendre and Gauss-Laguerre are two Guassian quadrature which, togheter with the well known Monte Carlo method, will be compared in accuaracy and speed for a multidimensional integral for a Helium atom.

The main idea of Gaussian quadrature is to integrate over a set of points $x_i$ not equally spaced with weights $w_i$. A part of the job is to find these points and weights(Program:Gauleg and Gauss Legendre). The weights are found throug ortogonal polynomials(Laguerre and Legendre polynomials) a set interval. The points $x_i$ are chosen in a optimal sense and lie in the interval.

Some theory is first presented with a following discussion of the three methodes mentioned above.

\end{document}
