\documentclass[../main.tex]{subfiles}

\begin{document}
\section{Introduction}\label{introduction}
The Verlet method is a widely used method for solving coupled ordinary differential equations. This method will be implemented  in order to make a simulation of our solarsystem. Due to several of these coupled ordinary differential equations it is nessecary to use object oriented code. The planets orbit is calculated using the Verlet method with different initial conditions for the different planets. This way of coding makes it easier to expand the algoritm if it is desireble to add more planets, moons and/or astronomical objects in the system. The equations used to calculate the movement of the planets is simply Newtons law of gravity and Newtons second law. Due to the enourmus mass of the Sun, its motion will be negligible compared with the other planets.\\
First we wil look at the discretized differential equations before making an algorithm to solve the Sun-Earth motion with both Euler's forward and the velocity Verlet method. Jupiter, the planet with the greates mass, has somewhat an inpact on the orbit of Earth, which we will try to figure out. Thereafter we will solve the same problem using object oriented code, thus for the whole solar system. The stability of our algorithm will be tested using different time steps $\Delta t$, as well as checking that the total energy(potential and kinetic) and the angular momentum is conserved. By trail end error we will try to figure out one of the planets escape velocity.
\end{document}
