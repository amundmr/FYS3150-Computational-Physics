\documentclass[../main.tex]{subfiles}
\begin{document}
\section{Results}\label{results}

%oppgave 5c)
\subsection{Testing}
\subsubsection{Stability with varying timestep} \label{sec:results-test-timestep}
In the figures below we plotted Earths orbit over a thousand years with different timesteps.

\begin{figure}[!h]
  \centering
  \makebox[\textwidth][c]{
  \begin{minipage}{0.7\textwidth}
        \centering
        \includegraphics[width=1\textwidth]{/test/Earth-Sun_Test0-01.png} % first figure itself
    \end{minipage}\hfill
    \begin{minipage}{0.7\textwidth}
        \centering
        \includegraphics[width=1\textwidth]{/test/Earth-Sun_Test0-02.png} % second figure itself
    \end{minipage}
}
  \caption{Earth orbit with time steps $\Delta t = 0.01$year and $0.02$year respectively }
  \label{fig:results-Timestep1}
\end{figure}
\FloatBarrier
\begin{figure}[!h]
  \centering
  \makebox[\textwidth][c]{
  \begin{minipage}{0.7\textwidth}
        \centering
        \includegraphics[width=1\textwidth]{/test/Earth-Sun_Test0-05.png} % first figure itself
    \end{minipage}\hfill
    \begin{minipage}{0.7\textwidth}
        \centering
        \includegraphics[width=1\textwidth]{/test/Earth-Sun_Test0-1.png} % second figure itself
    \end{minipage}
}
  \caption{Earth orbit with time steps $\Delta t = 0.05$year and $0.1$year respectively }
  \label{fig:results-Timestep2}
\end{figure}
\FloatBarrier

\subsubsection{Energy and angular momentum conservation}\label{sec:results-test-conservation}
In the figures below, kinetic energy and potential energy is plotted as a function of time in the system. We chose to simulate over a thousand years, with a timestep of $\Delta t = 0.01$.
\begin{figure}[!h]
  \centering
  \includegraphics[width=1\textwidth]{/test/Energy_Test.png} % first figure itself
  \caption{Kinetic and Potential energy with timestep $\Delta t = 0.01$year.}
  \label{fig:results-Energies}
\end{figure}
\FloatBarrier

%oppgave 5f)
Sammenlign resultetene med de tidligere.


\end{document}
