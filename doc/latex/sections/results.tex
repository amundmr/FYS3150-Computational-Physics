\documentclass[../main.tex]{subfiles}
\begin{document}
\section{Results}
\subsection{$2 \times 2$ lattice, analytical expressoins}
If we scale the value of $\beta$ from $1/k_BT$ to $1/J$ (Scaling factor $k_B T/J$) in the analytical expression from section \ref{sec:theory-analy}, we will get a good benchmark for computer computations to come. These values are listed in table \ref{tab:2x2spinsEnergiesMags} below. Note that all values are divided by four, since we want the values per bond, and not for the entire lattice.
\begin{table}[!h]
\begin{center}
  \begin{tabular}{l r}
    Mean energy, $\langle E \rangle$: & $-1.9960$  \\
    Mean absolute magnetization, $\langle |\mathcal{M}| \rangle$: & $0.9987$ \\
    Specific heat capacity, $C_V$: & $0.0321$\\
    Susceptibility, $\chi$: & $3.9933$
  \end{tabular}
  \caption{Benchmark for material characteristics per bond for a $2 \times 2$ lattice}
  \label{tab:2x2spinsEnergiesMags}
\end{center}
\end{table}
\FloatBarrier

\subsection{Ising model: simulation over temperature}
We ran the program for different amounts of Monte Carlo cycles and plottet the error (analytical - simulated) in figure \ref{fig:results-MCplot} below. It seems we want to use around $10^{7}$ MC cycles or more to get a good simulation.

\begin{figure}[!h]
\includegraphics[scale=0.7]{CyclesComparison.png}
\caption{Shows the accuracy of different amount of MC cycles over temperature.}
\label{fig:results-MCplot}
\end{figure}
\FloatBarrier

\subsection{$20 \times 20$ lattice, analytical expressoins}
T = 1.0(kT/J):
Mean energy and magnetization func of MC cycles:
Ordnet orientering: Program initilize.cpp (For T < 1.5 så er alle spinn opp,
ellers spinn ned)

sett inn følgende bilder fra mappe M+E under img:
T1_1.png
T1_2.png
Tekst: ordnet spinn orientering for T= 1.0

Random spinn orientering: Program initilize_random (Setter spinn ned(-1)
hvis verdien vi får mellom 0 og 1 er mindre eller lik 0.5 ).

sett inn følgende bilder fra mappe M+E under img:
L20T1random_1.png
L20T1random_2.png
Tekst: Random spinn orientering for T = 1.0


Likevekt ved:

T = 2.4(kT/J):
Mean energy and magnetization func of MC cycles:
Ordnet orientering: Program initilize.cpp (For T < 1.5 så er alle spinn opp,
ellers spinn ned)

sett inn følgende bilder fra mappe M+E(ligger inni img):
T2_1.png
T2_2.png
Tekst: ordnet spinn orientering for T= 2.4

Random spinn orientering: Program initilize_random (Setter spinn ned(-1)
hvis verdien vi får mellom 0 og 1 er mindre eller lik 0.5 ).

L20T1random_1.png
L20T1random_2.png
Tekst: Random spinn orientering for T = 2.4


Likevekt ved:

Oversiktelig tabell med når likevekt nås ca. (antall mcs)

    Ordnet magnetisering  Random magnetisering  Ordnet energi Random energi
T1:
T2:

Estimat av equilibration time:


Antall aksepterte spinn totalt etter et gitt antall mcs(100k maks):
Set start point
T = 1
Bilde :accepted_spinn_T1_mcs_cumsum(y)_log10.png
Stabiliserer seg ved mcd = 1E3.5(alle spinn blir heretter akseptert)

T = 2.4
Bilde: accepted_spinn_T2_mcs_cumsum(y)_log10.png
Stabiliserer seg ved mcd = 1E3.5, men det er mange flere som blir akseptert(Se y aksen)



Random start point:
T = 1:
Bilde: accepted_spins_T1_random_cumsum(y)_mcs_log10.png
T = 2.4
Bilde: accepted_spins_T2_random_cumsum(y)_mcs_log10.png

Temperaturavhengighet(skal vi lage plot her også- eller holder det med kommentar i resultater?):
Økt temperatur gjør at mange flere spinn aksepteres ved lavere antall mcs dvs tidligere.


Diskusjon/resultater
Aksepterte spinn som funksjon av T:
Økt temperatur gir flere aksepterte flips.
Setter man startpoint til random går den fortere mot likevekt ?


\end{document}
