\documentclass[../main.tex]{subfiles}
\begin{document}
\section{Results}
%---------------------------------------------------
\subsection{$2 \times 2$ lattice, analytical expressions}
If we scale the value of $\beta$ from $1/k_BT$ to $1/J$ (Scaling factor $k_B T/J$) in the analytical expression from section \ref{sec:theory-analy}, we will get a good benchmark for computer computations to come. These values are listed in table \ref{tab:2x2spinsEnergiesMags} below. Note that all values are divided by four, since we want the values per bond, and not for the entire lattice.
\begin{table}[!h]
\begin{center}
  \begin{tabular}{| l | r |}
    \hline
    \textbf{Mean energy,} $\mathbf{\langle E \rangle}$ & $-1.9960$  \\
    \hline
    \textbf{Mean absolute magnetization,} $\mathbf{\langle |\mathcal{M}| \rangle}$ & $0.9987$ \\
    \hline
    \textbf{Specific heat capacity,} $\mathbf{C_V}$ & $0.0321$\\
    \hline
    \textbf{Susceptibility,} $\mathbf \chi$ & $3.9933$ \\
    \hline
  \end{tabular}
  \caption{Benchmark for material characteristics per bond for a $2 \times 2$ lattice}
  \label{tab:2x2spinsEnergiesMags}
\end{center}
\end{table}
\FloatBarrier

%---------------------------------------------------
\subsection{Ising model: simulation over temperature} \label{sec:res-compareanalytical}
We ran the program for different amounts of Monte Carlo cycles and plotted the error ($\text{analytical} - \text{simulated}$) in figure \ref{fig:results-MCplot} below. Using $10^7$ Monte Carlo cycles, we seem to be getting pretty accurate results.

\begin{figure}[!h]
  \includegraphics[scale=0.7]{CyclesComparison.png}
  \caption{Shows the accuracy of different amount of MC cycles over temperature.}
  \label{fig:results-MCplot}
\end{figure}
\FloatBarrier
This shows that our computed results are quite close to our analytical results for the $20\times 20$ lattice. This is a good indication of a successfull simulation.
%---------------------------------------------------
\subsection{$20 \times 20$ lattice}
%---------------------------------------------------
\subsubsection*{Ordered spin orientation}
Initializing the spin structure, we first set every spin up for $T<1.5$ and every spin down for $T\ge 1.5$. In figure \ref{fig:ordered}, the computed values for the mean magnetization and energy are plotted against the number of MC cycles, at $T=1.0$ and $T=2.4$:

\begin{figure}[!h]
  \centering
  \makebox[\textwidth][c]{\includegraphics[width=1.3\textwidth]{./M+E/ordered.png}}
  \caption{Shows the computed value for the mean magnetization and energy, with ordered initialization, against the number of MC cycles. The scaled temperature is $T=1.0$ and $T=2.4$ respectively.}
  \label{fig:ordered}
\end{figure}
\FloatBarrier
All the plots pretty much stabilize into a value after 8000-1000 MC cycles. For $T=1.0$, the magnetization stabilizes around the value $0.99950$ and the energy around the value $-1.997$. This corresponds pretty good with the analytically calculated values. For $T=2.4$, the magnetization stabilizes around the value $0.5$ and the energy around the value $-1.25$.

%---------------------------------------------------
\subsubsection*{Random spin orientation}
Following the ordered initialisation, we also initialized the crystal randomly. In figure \ref{fig:random}, the computed values for the mean magnetization and energy are plotted against the number of MC cycles, at $T=1.0$ and $T=2.4$:

\begin{figure}[!h]
  \centering
  \makebox[\textwidth][c]{\includegraphics[width=1.3\textwidth]{./M+E/random.png}}
  \caption{Shows the computed value for the mean magnetization and energy, with random initialization, against the number of MC cycles. The scaled temperature is $T=1.0$ and $T=2.4$ respectively.}
  \label{fig:random}
\end{figure}
\FloatBarrier
The plots for $T=1.0$ follow a clean exponential curve, while the other plots pretty much stabilize after 8000-1000 MC cycles - like the previous ones. For $T=1.0$, the magnetization ends on the value $1.0$ and the energy on the value $-2.0$. This is similar to the analytical values, but does not have the same accuracy. For $T=2.4$, the magnetization stabilizes around the value $0.45$ and the energy around the value $-1.25$.
% Tekst: Random spinn orientering for T = 1.0

% Likevekt ved:
% T = 2.4(kT/J):
% Mean energy and magnetization func of MC cycles:
% Ordnet orientering: Program initilize.cpp (For T < 1.5 så er alle spinn opp,
% ellers spinn ned)
% Tekst: ordnet spinn orientering for T= 2.4
% Random spinn orientering: Program initilize_random (Setter spinn ned(-1)
% hvis verdien vi får mellom 0 og 1 er mindre eller lik 0.5 ).
% Tekst: Random spinn orientering for T = 2.4


% Likevekt ved:

% Oversiktelig tabell med når likevekt nås ca. (antall mcs)

%     Ordnet magnetisering  Random magnetisering  Ordnet energi Random energi
% T1:
% T2:

% Estimat av equilibration time:


% Antall aksepterte spinn totalt etter et gitt antall mcs(100k maks):
% Set start point
% T = 1

% Bilde :accepted_spinn_T1_mcs_cumsum(y)_log10.png
% Stabiliserer seg ved mcd = 1E3.5(alle spinn blir heretter akseptert)
%
% T = 2.4
% Bilde: accepted_spinn_T2_mcs_cumsum(y)_log10.png
% Stabiliserer seg ved mcd = 1E3.5, men det er mange flere som blir akseptert(Se y aksen)
%
%
%
% Random start point:
% T = 1:
% Bilde: accepted_spins_T1_random_cumsum(y)_mcs_log10.png
% T = 2.4
% Bilde: accepted_spins_T2_random_cumsum(y)_mcs_log10.png
%
% Temperaturavhengighet(skal vi lage plot her også- eller holder det med kommentar i resultater?):
% Økt temperatur gjør at mange flere spinn aksepteres ved lavere antall mcs dvs tidligere.(sjekk prosenten på y aksen)
% Ved random vs ikke random: omtrent like mange som aksepteres, men i random så aksepteres flere spinn ved lavere mcs. Ved T1 random får man en liten økning ved 1E1.5, mens hos T1 set startpont så før vi ikke en økning i aksepterte spinn før ved 1E3.5
%
% For T2 så får vi økningen på samme sted dvs 1E4 på både random og satt startpunkt.

% Diskusjon/resultater
% Aksepterte spinn som funksjon av T:
% Økt temperatur gir flere aksepterte flips.
% Setter man startpoint til random går den fortere mot likevekt ?

\subsection{Analyzing the probability distribution}
In figure \ref{fig:results-ProbE_T24} you can see the probability distribution for low and high temperature respectively. We can see that for a low temperature, the system tends to settle in the lowest energy state, while for the higher temperature the energies are a bit more spread. In table \ref{tab:results-variance} you will see the computed variance. Note that we calculated standard deviation of the histogram with \textsc{numpy.std}, and took the square root of this to get the variance.
\begin{figure}[!h]
  \centering
  \begin{minipage}{0.5\textwidth}
        \centering
        \includegraphics[width=1\textwidth]{ProbE_T1.png} % first figure itself
    \end{minipage}\hfill
    \begin{minipage}{0.5\textwidth}
        \centering
        \includegraphics[width=1\textwidth]{ProbE_T24.png} % second figure itself
    \end{minipage}

  \caption{Shows probability distribution for low and high temperature.}
  \label{fig:results-ProbE_T24}
\end{figure}
\FloatBarrier


\begin{table}[!h]
  \begin{center}
    \begin{tabular}{|l| l| l| l|}
      \hline
      Temperature & calculated variance & from histogram & deviation\\
      \hline
      1 & 9.375 & 1.752 & 81\%\\
      2.4 & 8.053  & 7.550 & 6.2\%\\
      \hline
    \end{tabular}
    \caption{Computed variance}
    \label{tab:results-variance}
  \end{center}
\end{table}
\FloatBarrier

\end{document}
