\documentclass[../main.tex]{subfiles}
\begin{document}
\section{Theory}
%---------------------------------------------
\subsection{The problem}
WRITE a bit about the system we want to solve! What is it?

Study a ferromagnetic matrix/crystal with one electronspinn at each lattice site. This is has been modeled as a L$\times$L matrix, where L is the number spinns in one direction. Initilizing both random and not(all spinns the same) to study when the most likely state/equlibrium is reached. Also studied the temperature dependence and have tried to find the critical temperature where ferromagnetism is lost.
Methods used are Monte carlo and the Ising model- meaning we could study a $100 \times 100$ lattice without it taking too much time.

Using boundary conditions we assume that the lattice is infinite in all directions. By simply setting the spinn on one edge to the same as the spinn of the neighbour to the spinn on the opposite side, the boundary will not affect the lattice.

%---------------------------------------------
\subsection{$2 \times 2$ lattice, analytical expressions} \label{sec:theory-analy}
To get started we will find the analytical expression for the partition function and the corresponding expectation values for the energy $E$, the mean absolute value of the magnetic moment $|M|$ (which we will refer to as magnetization), the specific heat $C_V$ and the susceptibility $\chi$ as function of T using periodic boundary conditions. These calculations will serve as benchmarks for our next steps.

%---------------------------------------------
\subsubsection*{Partition function, $Z$}
The partition function in the canonical ensemble is defined as:
\[ Z = \sum_{i=1}^M e^{-\beta E_i}\]
Where $\beta =\frac{1}{k_B T}$ and $E_i$ is the energy of the system in the microstate $i$ and $M$ is the number of microstates ($=2^N$ if $N$ is number of electrons).
\\
We therefore have to find $E_i$ which is defined as:
\[E_i = -J \sum_{<kl>}^N s_k s_l\]
Where $<kl>$ indicates that we sum only over the nearest neighbors and $J$ is a constant for the bonding strenght. For our two dimensional system the equation reads:
\[E_{i,2D} = -J \sum_i^N \sum_j^N \left(s_{i,j}s_{i,j+1} + s_{i,j}s_{i+1,j}\right)\]
Four our two-spin-state system with two dimensions we get the following table if we use periodic boundary conditions:
\begin{table}[!h]
  \begin{center}
    \begin{tabular}{| c | c | c | c |}
      \hline
      Number of spins up & Degeneracy & Energy & Magnetization\\
      \hline
      4 & 1 & -8J & 4\\
      3 & 4 & 0 & 2 \\
      2 & 4 & 0 & 0\\
      2 & 2 & 8J & 0 \\
      1 & 4 & 0 & -2 \\
      0 & 1 & -8J & -4 \\
      \hline
    \end{tabular}
    \caption{Number of spins up, degeneracy, energy and magnetization of the two-dimensional benchmark scenario.}
    \label{tab:2x2spinsEnergiesMags}
  \end{center}
\end{table}
\FloatBarrier
Where the magnetization is found by subtracting the number of spins down from the number of spins up, or in other words the sum of the spins:
\[\mathcal{M} = \sum_{j=1}^N s_j\]
Getting back to the partition function, we insert all $16$ of the $E_i$ respectively. For the degeneracies, we just multiply one iteration of the respecitve $E_i$ with the amount of degeneracies. When the energy $E_i$ is zero, we will just add one to the sum since $e^0 = 1$. Thus we get the following:

\[Z = e^{-\beta (-8J)} + 2 \cdot e^{-\beta (8J)} + e^{-\beta (-8J)} + 12 = 2e^{-\beta 8J} + 2e^{\beta 8J} + 12\]
\[Z = 4\cosh(\beta 8J) + 12\]

%---------------------------------------------
\subsubsection*{Energy expectation value, $\langle E \rangle$}
The expectation value of the energy is defined as:

\[\langle E \rangle = \sum_{i=1}^M E_i P_i(\beta) = \frac{1}{Z}\sum_{i=1}^M E_i e^{-\beta E_i}\]
Where $M$ is the sum over all microstates. $P_i$ is the Boltzmann probability distribution which reads:

\[P_i(\beta) = \frac{e^{-\beta E_i}}{Z}\]
For our system, this is easily calculated by inserting the partition function and the microstate energy $E_i$. The mean energy is then (calculations are shown in appendix, equation (\ref{eq:mean_energy})):

\begin{equation*}
  \langle E \rangle = -8J\frac{\sinh(\beta 8 J)}{\cosh(\beta 8 J) + 3}
\end{equation*}
Since the variance of the mean energy ($\sigma_E$) is needed for the heat capacity later, we will calculate this as well. Full calculation is found in the appendix, equation (\ref{eq:mean_energy_var}).

\begin{equation*}
  \sigma_E^2 = 64J^2\left(\frac{\cosh(\beta 8J)}{\cosh(\beta 8 J) + 3} - \left(\frac{\sinh(\beta 8 J)}{\cosh(\beta 8 J) + 3}\right)^2\right)
\end{equation*}

%---------------------------------------------
\subsubsection*{Magnetization expectation value, $\mathcal{M}$}
In the canonical ensemble the mean absolute magnetization can be described as
\[\langle |\mathcal{M}| \rangle = \sum_i^M |\mathcal{M}_i| P_i(\beta) = \frac{1}{Z} \sum_i^M |\mathcal{M}_i| e^{-\beta E_i}\]
We can now simply insert the magnetization and the energies for each respective microstate. This is found in table \ref{tab:2x2spinsEnergiesMags}. Using this, we find (shown in the appendix, equation (\ref{eq:mean_abs_mag})):

\begin{equation*}
  \langle \mathcal{|M|} \rangle = \frac{ 2e^{\beta 8J} + 4}{\cosh(\beta 8J) + 3}
\end{equation*}
Since the variance of the mean magnetization ($\sigma_M$) is needed for the susceptibility later, we will calculate this here. For this we will need the mean magnetization square $\langle \mathcal{M}^2 \rangle$, and the mean magnetization $\langle \mathcal{M} \rangle$. $\langle \mathcal{M} \rangle$ is shown to be $0$ in the appendix, equation (\ref{eq:mean_mag}), and  $\langle \mathcal{M}^2 \rangle = \frac{8 e^{\beta8J} +  8}{\cosh(\beta 8J) + 3}$ (shown in the appendix, equation (\ref{eq:mean_mag_square})). Thus, the variance is:
\begin{equation*}
  \sigma^2_\mathcal{M} = \langle \mathcal{M}^2 \rangle - \langle \mathcal{M} \rangle ^2 = \frac{8 e^{\beta8J} +  8}{\cosh(\beta 8J) + 3}
\end{equation*}

%---------------------------------------------
\subsubsection*{Specific heat capacity, $C_V$}
The specific heat capacity is defined as
\[C_V = \frac{\sigma_E^2}{k_B T^2}\]
Insering the value $\sigma^2_E$ we get
\[C_V = \frac{1}{k_B T^2}64J^2\left(\frac{\cosh(\beta 8 J)}{\cosh(\beta 8 J) + 3} -\left(\frac{-\sinh(\beta 8 J)}{\cosh(\beta 8 J) + 3}\right)^2  \right)\]
This is the main function we will be comparing to the values from our computations later.

\subsubsection*{Susceptibility, $\chi$}
The susceptibility is defined as
\[\chi = \frac{\sigma^2_\mathcal{M}}{k_B T^2}\]
Insering the value of $\sigma^2_\mathcal{M}$, we get
\[\chi = \frac{1}{k_B T^2}\frac{8 e^{\beta8J} +  8}{\cosh(\beta 8J) + 3}\]

\noindent Note that all the four abovementioned characteristics ($\langle E\rangle$, $\langle |\mathcal M|\rangle$, $C_V$ and $\chi$) are temperature dependent, through the variable $\beta = \frac{1}{k_B T}$. \cite{Mortenstatphys2019}.

%---------------------------------------------
\subsection{Ising model}
The Ising model is applied for the study of phase transistions at finite temperatures
for magnetic systems. Energy is expressed as:

\begin{equation}
  E = -J \sum  _{<kl>}^N s_ks_l \qquad s_k = \pm 1
\end{equation}
$N$ is the number of spins and $J$ is a constant expressing the interaction between neighboring spins. The sum is over the nearest neighbours only, indicated by $<kl>$ the above equation. For $J>0$ it is energetically favorable for neighboring spins to align. Leading to, at low temperatures, $T$, spontaneous magnetization.\\
A probability distribution is needed in order to calculate the mean energy $\langle E \rangle$ and magnetization $\langle \mathcal M \rangle$ at a given temperature. The distribution is given by:

\begin{equation}
  P_i(\beta)=  \sum  _{i = 1}^M s_ks_l \exp{-\beta E_i},
\end{equation}
where $M$ is the number of microstates and $P_i$ is the probability of having the system in a state/configuration $i$.\\
We utilize the Metropolis algorithm, which checks if we get a lower energy for the system by flipping a spin. If that is the case, we flip the spin. This is repeated, in the hopes of it reaching the lowest state in total. \\
The pseudocode looks as follows:

\lstinputlisting[]{pseudocode}

To show that our code has good correspondence to analytical results, we will compare calculations for a $20\times 20$ lattice with the analytical result for the same lattice. This is shown in section \ref{sec:res-compareanalytical}.

%---------------------------------------------
\subsection{Equlibrium- energy minimun}
In order to find the equlibrium of the system($20\times 20$ lattice) we perform a study of time, corresponding to the number of Monte Carlo cycles, one needs to reach equlibrium state from where we can compute expectation values. Equlibrium is reached when energy and magnetisation- the thermodynamical quanteties - reach their stady states. Further it will not be energetically favorable to flip spins when reaching equlibrium. So the number of accepted spinns per Monte Carlo cycle is at this point expected to move towards a constant value.

We will graphically plot expectation values for energy and magnetisation as a function of Monte Carlo cycles for different temperatures, $T=1.0$ and $T= 2.4$ using both a random and a set starting point for the initial spinn configuration.

%---------------------------------------------
\subsection{Analyzing the probability distribution}
We will also compute the probability of the energy, $P(E)$ for the system with $L = 20$ with the temperatures $T= 1$ and $T= 2.4$. This is computed by counting the number of times a given energy appears in our computation. The computation will start at a number of Monte Carlo cycles of which we know that the system is stable. From the results section \ref{sec:results-20x20lattice}, we see that after the 10000th Monte Carlo cycle we are well beyond the stability limit, so this is what we will be using. We will also compare the width of the gaussian probability distribution (the standard deviation) with the computed variance energy $\sigma_E^2$ and discuss the behavior. The standard deviation and the variance should be connected through the fact that standard deviation is the square root of the variance.


%---------------------------------------------
\subsection{Numerical studies of phase transitions}
We wish to study the behavior of the Ising model in two dimensions close to the critical temperature as a function of the lattice size $L \times L$.  To do this, we will be calculating the expectation values for energy, $\langle E \rangle$, and absolute magnetisation, $\langle |M| \rangle$, and also the specific heat capacity $C_V$ and the susceptibility $\chi$ as a function of the temperature in the interval $ T \in [2.0,2.5]$ with a step of $\Delta T = 0.005$. This will be done for the lattice sizes $L = {20, 40, 60, 80, 100}$. From the plots we hope to see an indication of a phase transition.\\
The code will be paralellized using MPI, and timed using the \textsc{MPI\_Time()} function.


%---------------------------------------------
\subsection{Extracting the critical temperature}
We would like to compute the critical temperature $T_C$ in the thermodynamic limit where $L \rightarrow \infty$. This will be done using the equation below, together with the exact result of $\nu = 1$.
\[T_C(L) - T_C(L = \infty) = aL^{-1/\nu}\]
By using two different lattice sizes, we can calculate the factor a, and then calculate  $T_C$.
\begin{align}
  T_C(L_1) - T_C(L = \infty) &= aL_1^{-1/\nu} \nonumber\\
  T_C(L_2) - T_C(L = \infty) &= aL_2^{-1/\nu} \nonumber\\
  \intertext{By subtracting the equations we obtain}
  \frac{T_C(L_1) - T_C(L_2)}{L_1^{-1/\nu} - L_2^{-1/\nu}} &= a \label{eq:a}
\end{align}
It will now be easy to use the equation below to find the critical temperature for an infinitely large lattice.
\begin{align}
T_C(L = \infty) &= T_C(L) - aL^{-1/\nu} \label{eq:TC}
\end{align}


\end{document}
