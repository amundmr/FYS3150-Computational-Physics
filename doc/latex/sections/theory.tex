\documentclass[../main.tex]{subfiles}

\begin{document}
\section{Theory}\label{theory}
\subsection{The Earth-Sun system}
%oppgave 5a)
Before starting coding with object orientation, we will look at the problem by simply using Eulers forward method and something starting with V..
In two dimensions we will have the following for the Earth-Sun system \\

The gravitational force  $\text{F}_G$

\begin{equation}
  F = \frac{GM_0M_E}{r^2}
\end{equation}

where,

$M_E = 6\times 10^{24}\text{Kg},  M_0 = 2\times 10^{30}\text{Kg} \quad \text{and}\quad  r = 1.5\times 10^{11}\text{m}$\\

The force acting on Earth is given by Newtons 2. law, here given in x- and y- direction

$$ \frac{d^2x}{dt^2} = \frac{F_x}{M_E}, \quad \frac{d^2y}{dt^2} = \frac{F_y}{M_E}$$

by using the following equalities $$x = r \cos(\theta),\quad  y = r\sin(\theta), \quad \text{and}\quad r = \sqrt{x^2 +y^2}$$

we obtain\\

\begin{equation}
  F_x = - \frac{GM_0M_E}{r^2} \cos(\theta) =- \frac{GM_0M_E}{r^3}x
\end{equation}
\begin{equation}
  F_y = - \frac{GM_0M_E}{r^2}\sin(\theta) =- \frac{GM_0M_E}{r^3}y,
\end{equation}

This gives the following first order  coupled differential equations:
\begin{equation}
  \frac{dv_x}{dt} = - \frac{GM_0}{r^3}x
\end{equation}
\begin{equation}
  \frac{dx}{dt} = v_x
\end{equation}
\begin{equation}
  \frac{dv_y}{dt} = - \frac{GM_0}{r^3}y
\end{equation}
\begin{equation}
  \frac{dy}{dt} = v_y,
\end{equation}

In order to simplify we will use Astronomical units AU defined by $r$, which is the avarage distance between Earth and the sun.

Introducing astronomical units: 1 $\text{AU = r} = 1.5 \times 10^{11}$

\begin{equation}
  \frac{M_ev^2}{r} = F = \frac{GM_0M_E}{r^2}
\end{equation}

Since $GM_=  v^2r$ and the velocity of Earth, assuming it moves in circular motion: $$v = 2\pi \text{r/years}= 2\pi \text{AU/years}$$

Then we have the following relationship

$$GM_0 = v^2r = 4\pi^2 \frac{(\text{AU})^2}{\text{years}^2}$$

Bulding code- discretized equations:
$$v_{x,i+1} = v_{x,i} - h \frac{4\pi^2}{r_i^3}x_i$$
$$x_{i+1} = x_{i} + hv_{x,i}$$
$$v_{y,i+1} = v_{xy,i} - h \frac{4\pi^2}{r_i^3}y_i$$
$$y_{i+1} = y_{i} + hv_{y,i}$$

\subsection{The Verlet method}
Another numerical method to be used to evalute the motion of planets in our solarsystem is the Verlet method. This is a method pretty easy to implment as well as it gives stable results. When calculating molecular dynamics, this method is one of the first choises to implement.

If we again look at Newtons second law in the form of a second order differential equation in one dimension.

\begin{equation}
 m \frac{d^2x}{dt^2} = F(x,t)
\end{equation}

In coupled diffenretial equations one obtain

$$\frac{dx}{dt} = v(x,t)$$    and     $$\frac{dv}{dt} = F(x,t)/m = a(x,t)$$

Using a Taylor expansion:

\begin{equation}
 x(t,h) = x(t) + hc^{(1)}(t) + \frac{h^2}{2}x^{(2)}(t) + O(h^3).
\end{equation}

From Newtons second law we allready have obtained the second derivative, $x^{(2)}(t) = a(x,t)$.

Using Taylor for $x(t-h)$ and the discretized expressions $x(t_i,\pm h) = x_{i \pm 1}$ and $x_i = x(t_i)$ we obtain

\begin{equation}
  x_{i+1} = 2x_i - x_{i-1} + h^2 x_i^{(2)} + O(h^4)
\end{equation}

Corresponding velocity Taylor expansion is

\begin{equation}
  v_{i+1} = v_i + hv_{(1)} + \frac{h^2}{2} v_i^{(2)} + O(h^3)
\end{equation}

With Newtons second law:

\begin{equation}
  v_i^{(1)} = \frac{d^2x}{dt^2} = \frac{F(x_i, t_i)}{m},
\end{equation}

Adding the expansion of the derivative of the velocity

\begin{equation}
  v_{i+1} = v_i + \frac{h}{2}\left(v_{i+1}^{(1)} + v_i^{(1)}\right) + O(h^3)
\end{equation}

Since our error goes as $O(h^3)$ we only use the terms up to the second derivative of the velocity.

\begin{equation}
  hv_i^{(2)}\approx v_{i+1}^{(1)} - v_i^{(1)}

\end{equation}

Rewriting the Taylor expansions for the velocity:

\begin{equation}
  x_{i+1} = x_i + hv_i + \frac{h^2}{2} v_i^{(1)} + + O(h^3)
\end{equation}

and

\begin{equation}
  v_{i+1} = v_i + \frac{h}{2} \left(v_{i+1}^{(1)} + v_i^{(1)}\right) + O(h^3)
\end{equation}


\end{document}
