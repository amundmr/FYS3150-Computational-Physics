\documentclass[../main.tex]{subfiles}
\begin{document}
  \section{Theory}
  \subsection{The problem}
  \subsection{$2 \times 2$ lattice, analytical expressions}
    To get started we will find the analytical expression for the partition function and the corresponding expectation values for the energy $E$, the mean absolute value of the magnetic moment $|M|$ (which we will refer to as magnetization), the specific heat $C_V$ and the susceptibility $\chi$ as function of T using periodic boundary conditions. These calculations will serve as benchmarks for our next steps.

    \textbf{Partition function, $Z$}\\
    The partition function in the canonical ensemble is defined as:
    \[ Z = \sum_{i=1}^N e^{-\beta E_i}\]
    Where $\beta =\frac{1}{k_B T}$ and $E_i$ is the energy of the system in the microstate $i$ and $N$ is the respective microstate.
    \\
    We therefore have to find $E_i$ which is defined as:
    \[E_i = -J \sum_{<kl>}^N s_k s_l\]
    Where $<kl>$ indicates that we sum only over the nearest neighbors and J is a constant for the bonding strenght. For our two dimensional system the equation reads:
    \[E_{i,2D} = -J \sum_i^N \sum_j^N \left(s_{i,j}s_{i,j+1} + s_{i,j}s_{i+1,j}\right)\]
    Four our two-spin-state system with two dimensions we get the following table if we use periodic boundary conditions:
    \begin{table}[!h]
      \begin{tabular}{c c c c}
        Number of spins up & Degeneracy & Energy & Magnetization\\
        \hline\\
        4 & 1 & -8J & 4\\
        3 & 4 & 0 & 2 \\
        2 & 4 & 0 & 0\\
        2 & 2 & 8J & 0 \\
        1 & 4 & 0 & -2 \\
        0 & 1 & -8J & -4
      \end{tabular}
      \caption{Number of spins up, degeneracy, energy and magnetization of the two-dimensional benchmark scenario.}
    \end{table}
    \FloatBarrier
    Where the magnetization is found by subtracting the number of spin downs from the number of spin up, or in other words the sum of the spins:
    \[\mathcal{M} = \sum_{j=1}^N s_j\]

    Getting back to the partition function, we insert all $16$ of the $E_i$ respectively.

    \[Z = e^{-\beta (-8J)} + 2 \cdot e^{-\beta (8J)} + e^{-\beta (-8J)} = 2e^{-\beta 8J} + 2e^{\beta 8J}\]

    \textbf{Energy expectation value, $\langle E \rangle$}\\
    The expectation value of the energy is defined as:

    \[\langle E \rangle = \sum_{i=1}^M E_i P_i(\beta) = \frac{1}{Z}\sum_{i=1}^M E_i e^{-\beta E_i}\]

    Where $M$ is the sum over all microstates. $P_i$ is the Boltzmann probability distribution which reads:

    \[P_i(\beta) = \frac{e^{\beta E_i}}{Z}\]

    For our system, this is easily calculated by inserting the partition function and the microstate energy $E_i$.
    \begin{align*}
      \langle E \rangle &= \frac{1}{2e^{-\beta 8J} + 2e^{\beta 8J}} \left(2 \cdot -8J \cdot e^{\beta 8J} + 2\cdot 8J \cdot e^{-\beta8J}\right)\\
        &= \frac{1}{2e^{-\beta 8J} + 2e^{\beta 8J}} \left(-16J e^{\beta8J} + 16Je^{-\beta 8J}\right)\\
        &= 8J \frac{1}{e^{-\beta 8J} + e^{\beta 8J}}  \left(e^{-\beta8J} - e^{\beta 8J}\right)
    \end{align*}
    

\end{document}
