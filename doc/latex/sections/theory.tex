\documentclass[../main.tex]{subfiles}
\begin{document}
\section{Theory}
%---------------------------------------------
\subsection{The problem}
WRITE a bit about the system we want to solve! What is it?

%---------------------------------------------
\subsection{$2 \times 2$ lattice, analytical expressions} \label{sec:theory-analy}
To get started we will find the analytical expression for the partition function and the corresponding expectation values for the energy $E$, the mean absolute value of the magnetic moment $|M|$ (which we will refer to as magnetization), the specific heat $C_V$ and the susceptibility $\chi$ as function of T using periodic boundary conditions. These calculations will serve as benchmarks for our next steps.

%---------------------------------------------
\subsubsection*{Partition function, $Z$}
The partition function in the canonical ensemble is defined as:
\[ Z = \sum_{i=1}^M e^{-\beta E_i}\]
Where $\beta =\frac{1}{k_B T}$ and $E_i$ is the energy of the system in the microstate $i$ and $M$ is the number of microstates ($=2^N$ if $N$ is number of electrons).
\\
We therefore have to find $E_i$ which is defined as:
\[E_i = -J \sum_{<kl>}^N s_k s_l\]
Where $<kl>$ indicates that we sum only over the nearest neighbors and $J$ is a constant for the bonding strenght. For our two dimensional system the equation reads:
\[E_{i,2D} = -J \sum_i^N \sum_j^N \left(s_{i,j}s_{i,j+1} + s_{i,j}s_{i+1,j}\right)\]
Four our two-spin-state system with two dimensions we get the following table if we use periodic boundary conditions:
\begin{table}[!h]
  \begin{center}
    \begin{tabular}{| c | c | c | c |}
      \hline
      Number of spins up & Degeneracy & Energy & Magnetization\\
      \hline
      4 & 1 & -8J & 4\\
      3 & 4 & 0 & 2 \\
      2 & 4 & 0 & 0\\
      2 & 2 & 8J & 0 \\
      1 & 4 & 0 & -2 \\
      0 & 1 & -8J & -4 \\
      \hline
    \end{tabular}
    \caption{Number of spins up, degeneracy, energy and magnetization of the two-dimensional benchmark scenario.}
    \label{tab:2x2spinsEnergiesMags}
  \end{center}
\end{table}
\FloatBarrier
Where the magnetization is found by subtracting the number of spins down from the number of spins up, or in other words the sum of the spins:
\[\mathcal{M} = \sum_{j=1}^N s_j\]
Getting back to the partition function, we insert all $16$ of the $E_i$ respectively. For the degeneracies, we just multiply one iteration of the respecitve $E_i$ with the amount of degeneracies. When the energy $E_i$ is zero, we will just add one to the sum since $e^0 = 1$. Thus we get the following:

\[Z = e^{-\beta (-8J)} + 2 \cdot e^{-\beta (8J)} + e^{-\beta (-8J)} + 12 = 2e^{-\beta 8J} + 2e^{\beta 8J} + 12\]
\[Z = 4\cosh(\beta 8J) + 12\]

%---------------------------------------------
\subsubsection*{Energy expectation value, $\langle E \rangle$}
The expectation value of the energy is defined as:

\[\langle E \rangle = \sum_{i=1}^M E_i P_i(\beta) = \frac{1}{Z}\sum_{i=1}^M E_i e^{-\beta E_i}\]
Where $M$ is the sum over all microstates. $P_i$ is the Boltzmann probability distribution which reads:

\[P_i(\beta) = \frac{e^{-\beta E_i}}{Z}\]
For our system, this is easily calculated by inserting the partition function and the microstate energy $E_i$. The mean energy is then (calculations are shown in appendix, equation (\ref{eq:mean_energy})):

\begin{equation*}
  \langle E \rangle = -8J\frac{\sinh(\beta 8 J)}{\cosh(\beta 8 J) + 3}
\end{equation*}
Since the variance of the mean energy ($\sigma_E$) is needed for the heat capacity later, we will calculate this as well. Full calculation is found in the appendix, equation (\ref{eq:mean_energy_var}).

\begin{equation*}
  \sigma_E^2 = 64J^2\left(\frac{\cosh(\beta 8J)}{\cosh(\beta 8 J) + 3} - \left(\frac{\sinh(\beta 8 J)}{\cosh(\beta 8 J) + 3}\right)^2\right)
\end{equation*}

%---------------------------------------------
\subsubsection*{Magnetization expectation value, $\mathcal{M}$}
In the canonical ensemble the mean absolute magnetization can be described as
\[\langle |\mathcal{M}| \rangle = \sum_i^M |\mathcal{M}_i| P_i(\beta) = \frac{1}{Z} \sum_i^M |\mathcal{M}_i| e^{-\beta E_i}\]
We can now simply insert the magnetization and the energies for each respective microstate. This is found in table \ref{tab:2x2spinsEnergiesMags}. Using this, we find (shown in the appendix, equation (\ref{eq:mean_abs_mag})):

\begin{equation*}
  \langle \mathcal{|M|} \rangle = \frac{ 2e^{\beta 8J} + 4}{\cosh(\beta 8J) + 3}
\end{equation*}
Since the variance of the mean magnetization ($\sigma_M$) is needed for the susceptibility later, we will calculate this here. For this we will need the mean magnetization square $\langle \mathcal{M}^2 \rangle$, and the mean magnetization $\langle \mathcal{M} \rangle$. $\langle \mathcal{M} \rangle$ is shown to be $0$ in the appendix, equation (\ref{eq:mean_mag}), and  $\langle \mathcal{M}^2 \rangle = \frac{8 e^{\beta8J} +  8}{\cosh(\beta 8J) + 3}$ (shown in the appendix, equation (\ref{eq:mean_mag_square})). Thus, the variance is:
\begin{equation*}
  \sigma^2_\mathcal{M} = \langle \mathcal{M}^2 \rangle - \langle \mathcal{M} \rangle ^2 = \frac{8 e^{\beta8J} +  8}{\cosh(\beta 8J) + 3}
\end{equation*}

%---------------------------------------------
\subsubsection*{Specific heat capacity, $C_V$}
The specific heat capacity is defined as
\[C_V = \frac{\sigma_E^2}{k_B T^2}\]
Insering the value $\sigma^2_E$ we get
\[C_V = \frac{1}{k_B T^2}64J^2\left(\frac{\cosh(\beta 8 J)}{\cosh(\beta 8 J) + 3} -\left(\frac{-\sinh(\beta 8 J)}{\cosh(\beta 8 J) + 3}\right)^2  \right)\]
This is the main function we will be comparing to the values from our computations later.

\subsubsection*{Susceptibility, $\chi$}
The susceptibility is defined as
\[\chi = \frac{\sigma^2_\mathcal{M}}{k_B T^2}\]
Insering the value of $\sigma^2_\mathcal{M}$, we get
\[\chi = \frac{1}{k_B T^2}\frac{8 e^{\beta8J} +  8}{\cosh(\beta 8J) + 3}\]

\noindent Note that all the four abovementioned characteristics ($\langle E\rangle$, $\langle |\mathcal M|\rangle$, $C_V$ and $\chi$) are temperature dependent, through the variable $\beta = \frac{1}{k_B T}$. \cite{Mortenstatphys2019}.

%---------------------------------------------
\subsection{Ising model}
The Ising model is applied for the study of phase transistions at finite temperatures
for magnetic systems. Energy is expressed as:

\begin{equation}
  E = -J \sum  _{<kl>}^N s_ks_l \qquad s_k = \pm 1
\end{equation}
$N$ is the number of spins and $J$ is a constant expressing the interaction between neighboring spins. The sum is over the nearest neighbours only, indicated by $<kl>$ the above equation. For $J>0$ it is energetically favorable for neighboring spins to align. Leading to, at low temperatures, $T$, spontaneous magnetization.\\
A probability distribution is needed in order to calculate the mean energy $\langle E \rangle$ and magnetization $\langle \mathcal M \rangle$ at a given temperature. The distribution is given by:

\begin{equation}
  P_i(\beta)=  \sum  _{i = 1}^M s_ks_l \exp{-\beta E_i},
\end{equation}
where $M$ is the number of microstates and $P_i$ is the probability of having the system in a state/configuration $i$.\\
We utilize the Metropolis algorithm, which checks if we get a lower energy for the system by flipping a spin. If that is the case, we flip the spin. This is repeated, in the hopes of it reaching the lowest state in total. \\
The pseudocode looks as follows:

\lstinputlisting[]{pseudocode}

%---------------------------------------------
\subsection{Analyzing the probability distribution}
We will also compute the probability of the energy, $P(E)$ for the system with $L = 20$ with the temperatures $T= 1, 2.4$. This is computed by counting the number of times a given energy appears in our computation. The computation will start at a number of Monte Carlo cycles of which we know that the system is stable. We will also compare the results with the computed variance energy $\sigma_E^2$ and discuss the behavior.

\end{document}
