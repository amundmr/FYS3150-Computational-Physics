\documentclass[../main.tex]{subfiles}
\begin{document}
\section{Theory}
\subsection{Ising model}
The Ising model is applied for the study of phase transistions at finite temperatures
for magnetic systems. Energy is expressed as:

\begin{equation}
  E = -J \sum  _{<kl>}^N s_ks_l \qquad s_k = \pm 1
\end{equation}

N is the number of spins and J is a constant expressing the interaction between neighboring spins. The sum is over the nearest neighbours only, indicated by <kl> the above equation. For J > 0 it is energetically favorable for neighboring spins to align. Leading to, at low temperatures, T, spontanious magnetisation.

A probability distribution is neede in order to calculate the mean energy <E> and magnetization <M> at a given temperature. The distribution is given by:

\begin{equation}
  P_i(\beta)=  \sum  _{i = 1}^M s_ks_l \exp{-\beta E_i},
\end{equation}

where M is all the microstates, $P_i$ is the probability of having the system in a state/configuration i.

\end{document}
