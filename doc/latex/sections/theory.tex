\documentclass[../main.tex]{subfiles}
\begin{document}
  \section{Theory}
  \subsection{The problem}
  \subsection{$2 \times 2$ lattice, analytical expressions}
    To get started we will find the analytical expression for the partition function and the corresponding expectation values for the energy $E$, the mean absolute value of the magnetic moment $|M|$ (which we will refer to as magnetization), the specific heat $C_V$ and the susceptibility $\chi$ as function of T using periodic boundary conditions. These calculations will serve as benchmarks for our next steps.

    \textbf{Partition function}\\
    The partition function in the canonical ensemble is defined as:
    \[ Z = \sum_{i=1}^N e^{-\beta E_i}\]
    Where $\beta =\frac{1}{k_B T}$ and $E_i$ is the energy of the system in the microstate $i$ and $N$ is the respective microstate.
    \\
    We therefore have to find $E_i$ which is defined as:
    \[E_i = -J \sum_{<kl>}^N s_k s_l\]
    Where $<kl>$ indicates that we sum only over the nearest neighbors and J is a constant for the bonding strenght. For our two dimensional system the equation reads:
    \[E_{i,2D} = -J \sum_i^N \sum_j^N \left(s_{i,j}s_{i,j+1} + s_{i,j}s_{i+1,j}\right)\]
    Four our two-spin-state system with two dimensions we get the following table:
    \begin{table}[!h]
      \begin{tabular}{c c c c}
        Number of spins up & Degeneracy & Energy & Magnetization\\
        \hline\\
        4 & 1 & -8J & 4\\
        3 & 4 & 0 & 2 \\
        2 & 4 & 0 & 0\\
        2 & 2 & 8J & 0 \\
        1 & 4 & 0 & -2 \\
        0 & 1 & -8J & -4
      \end{tabular}
      \caption{Number of spins up, degeneracy, energy and magnetization of the two-dimensional benchmark scenario.}
    \end{table}
    \FloatBarrier
    Where the magnetization is found by subtracting the number of spin downs from the number of spin up.


    \subsection{Ising model}
    The Ising model is applied for the study of phase transistions at finite temperatures
    for magnetic systems. Energy is expressed as:

    \begin{equation}
      E = -J \sum  _{<kl>}^N s_ks_l \qquad s_k = \pm 1
    \end{equation}

    N is the number of spins and J is a constant expressing the interaction between neighboring spins. The sum is over the nearest neighbours only, indicated by <kl> the above equation. For J > 0 it is energetically favorable for neighboring spins to align. Leading to, at low temperatures, T, spontanious magnetisation.

    A probability distribution is neede in order to calculate the mean energy <E> and magnetization <M> at a given temperature. The distribution is given by:

    \begin{equation}
      P_i(\beta)=  \sum  _{i = 1}^M s_ks_l \exp{-\beta E_i},
    \end{equation}

    where M is all the microstates, $P_i$ is the probability of having the system in a state/configuration i.

    CONFIGURATIONS




\end{document}
