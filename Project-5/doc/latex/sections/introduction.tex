\documentclass[../main.tex]{subfiles}

\begin{document}
\section{Introduction}\label{introduction}
The Velocity Verlet method is a widely used method for solving coupled ordinary differential equations. In this report, we will model our solar system's dynamics, utilizing said method. The equations to solve simply come from Newton's laws of motion in gravitational fields, although we will make a small modification down the road to account for relativistic effects and achieve greater accuracy.\\
Due to the sheer number of variables and methods required to calculate the motion of this many bodies, we will object orient our code - simplifying the process of adding bodies and making them interact with each other. \\
On our way to the final model of the solar system, we are going to explore the accuracy differences between the Velocity Verlet and the Euler Forward method, as well as which timestep is needed for sufficient accuracy. The last check before we get going is to make sure that energy is conserved in the system. Thereafter we investigate different aspects of the solar system with our model. This includes what impact the massive planet Jupiter might have on the orbit of Earth, and at what speed the perihelion of Mercury precesses around the sun, when relativistic effects are accounted for. \\
The report has a theory part explaining the physical theory and the thought behind our computational implementation of it. Following this are our results and finally a discussion of them.
% Might lack some parts. Just hmu on Messenger if something needs to be added.
\end{document}
