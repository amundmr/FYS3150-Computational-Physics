\documentclass{article}
\usepackage[utf8]{inputenc}
\usepackage{amsmath}
\usepackage{geometry}
\usepackage{dirtytalk}

\title{FYS3150 - Project 1}
\author{Aasrud, Raniseth, Rongve}
\date{\today}

\begin{document}

\maketitle

\section*{Project 1 a)}

For $i = 0$ and $i = n$ the boundary conditions gives us
$v(0) = v(1) = 0$.

For $i = 1$
\[-\frac{v_2+v_0-2v_1}{h^2} = f_1\]

For $i = 2$\\
\[-\frac{v_3+v_1-2v_2}{h^2}{} = f_2\]

For $i = n-1$
\[-\frac{v_n+v_{n-2}-2v_{n-1}}{h^2}{} = f_{n-1}\]

If you multiply both sides by $h^2$

\[-{v_2+v_0-2v_1} = h^2\cdot{f_1}\]

\[-{v_3+v_1-2v_2} = {h^2}\cdot{f_2}\]

\[-v_n+v_{n-2}-2v_{n-1} = {h^2}\cdot{f_{n-1}}\]

Which you can rewrite as a linear set of equations on the form $A\hat{v} = \tilde{b}_i$ where

\[ A = \begin{bmatrix}
2 & -1 & 0 & 0 & \hdots & 0\\
-1 & 2 & -1 & 0 & \hdots & 0\\
\vdots & \vdots & \vdots & \ddots & & \vdots\\
0 & 0 & 0 & \hdots & -1 & 2
 \end{bmatrix}\hspace{1cm}
 \hat{v} = \begin{bmatrix}
 v_1\\
v_2\\
\vdots\\
v_{n-1}
 \end{bmatrix}\hspace{1cm}
 \tilde{b_i} = \begin{bmatrix}
 \tilde{b_1}\\
\tilde{b_2}\\
\vdots\\
\tilde{b_{n-1}}
 \end{bmatrix}
 \]

And $\tilde{b_i} = h^2 \cdot f_i$

\section*{Project 1 b)}
Den generelle algoritmen blir som følgende:
\begin{align*}
  \tilde{c}[i] &= \frac{c[i]}{b[i]-a[i]*\tilde{c}[i-1]}\\
  \vspace{1cm}\\
  \tilde{d}[i] &= \frac{d[i]-a[i]*\tilde{d}[i-1]}{b[i]-a[i]*\tilde{c}[i-1]}
\end{align*}
Siden nevneren er lik i begge brøker kan vi regne ut den først for hver $i$.
Da får vi $6$ floating point operations per n, altså $6n$ FLOPS.

\end{document}
