\documentclass{article}
\usepackage{graphicx}
\usepackage[utf8]{inputenc}
\usepackage[fleqn]{amsmath}
\usepackage{titling}
\usepackage{graphicx,wrapfig,lipsum}
\usepackage{amssymb}
\usepackage{listings}
\usepackage[font=small,labelsep=none]{caption}

\setlength{\droptitle}{-10em}

\title{Project 1 FYS3150}\vspace{-3ex}
\author{Anna Stray Rongve}
\date{\vspace{-5ex}}

\begin{document}

\maketitle
\section*{Abstract}
\maketitle
\section*{Introduction}
\maketitle
\section*{Theory and technicalites}
\maketitle
\section*{Conclusion and perspectives}

\maketitle
\section*{Project 1 a)}

For $i = 0$ and $i = n$ the boundary conditions gives us $v(0) = v(1) = 0$.
\\
For $i = 1$

\begin{equation}
   -\frac{v_2+v_0-2v_1}{h^2} = f_1
\end{equation}
\\
For $i = 2$\\
\begin{equation}
   -\frac{v_3+v_1-2v_2}{h^2}{} = f_2
\end{equation}
\vdots\\
\\
For  $i = n-1$\\
\\
\begin{equation}
  -\frac{v_n+v_{n-2}-2v_{n-1}}{h^2}{} = f_{n-1}
\end{equation}\\
If you multiply both sides by $h^2$

\begin{equation}
   -{v_2+v_0-2v_1} = h^2\cdot{f_1}
\end{equation}
\\
\begin{equation}
   -{v_3+v_1-2v_2} = {h^2}\cdot{f_2}
\end{equation}
\vdots\\
\begin{equation}
  -v_n+v_{n-2}-2v_{n-1} = {h^2}\cdot{f_{n-1}}
\end{equation}
\\Which you can rewrite as a linear set of equations where\\

\\$A = $
 \begin{bmatrix}
  2 & -1 & 0 & \cdots & \cdots & 0 \\
  -1 & 2 & -1 & 0 & \cdots &\cdots\\
  0 & -1 &2 & -1 & 0 & \cdots \\
   & \cdots & \cdots & \cdots & \cdots & \cdots \\
   0 & \cdots & \cdots & -1 & 2 & -1\\
   0 & \cdots & \cdots & 0 & -1 & 2
\end{bmatrix}
\newpage
\]\\
$\hat{v}= $
 \begin{bmatrix}
  v_1 & v_2 & & \cdots & v_{n-1}\\
\end{bmatrix}
\]

\newline
and
\newline
$\tilde{b_i} = $
 \begin{bmatrix}
  \tilde{b_1} & \tilde{b_2} & & \cdots & \tilde{b_{n-1}}
\end{bmatrix}
\]

\\Where $ \tilde{b_i} = h^2 \cdot f_i $

\maketitle
\section*{Project 1 b)}

\section*{Appendix}
\maketitle
\section*{Bibliography}
\maketitle

\end{document}
